\section{Slab-column connection}
Tests on slab-column connections subjected to reversed cyclic loading\citep{carpenter1973design,symonds1976slab} indicate that ductility of flat slab-column connections can be significantly increased through the use of stirrups enclosing bands of flexural slab reinforcement passing through the columns\footnote{Such shear-reinforced bands would essentially function as shallow beams connecting the columns.}. \citep{Robertson2006} observed that an increase in the flexural stiffness of a slab-column connection induces a higher shear demand. Also \citep{Robertson2006} concluded that connections with increased slab flexural reinforcement will support greater lateral loads, but the increased eccentric shear transfer may result in premature punching shear failure. However tests on interior column-to-slab connections with lightly reinforced slabs with and without shear reinforcement \citep{peiris2012flexural,hawkins2017effect,bayrak2009two,muttoni2008punching,dam2017behavior,muttoni2008punching} have shown that yielding of slab flexural tension reinforcement in the vicinity of the column or loaded area leads to increased local rotations and opening of any inclined crack existing within the slab. In such cases, sliding along the inclined crack can cause a flexure-driven punching failure at a shear force less than the strength calculated by the two-way shear equations of \citet[Table 22.6.5.2]{aci31819} (\ref{eqt22652}) for slabs without shear reinforcement and less than the strength calculated in accordance with \citet[Section 22.6.6.3]{aci31819} for slabs with shear reinforcement. 

When ultimate shear exceeds slab capacity ($v_u>\phi v_n$), the slab shear capacity can be increased by:\begin{itemize}
\item Thickening the slab in the column vicinity implementing or adding:
\begin{itemize}
\item Column capital
\item Shear capital or drop panel;
\end{itemize}
\item Shear reinforcement;
\item Increasing specified concrete compressive strength $f'_c$;
\item Increasing column dimensions or support area.
\end{itemize}

Tests of slabs with flexural reinforcement less than $A_{s,\min{}}$ have shown that shear reinforcement does not increase the punching shear strength\citep{aci31819}. However, shear reinforcement may increase plastic rotations prior to the flexure-driven punching failure\citep{peiris2012flexural}. Whereas \cite{megally2000punching,kang2006,robertson2002cyclic} argue that connection ductility will increase with shear reinforcement application in slab-column connections. This is sound mainly because slab connection will yield or fail in flexure prior to punching shear. \cite{kang2006,megally2000punching,robertson1991,robertson1993,anggadjaja2008} showed that gravity load strongly influences flat slab-column connection lateral ductility and with gravity load increase connection lateral displacement capacity decreases that was also observed in reviewed test data by \cite{aci4212010} to assess gravity load effect on lateral drift capacity for interior flat plate-column connection specimens with and without shear reinforcement. Also tests have shown that beam-column joints laterally supported on four sides by beams of approximately equal depth exhibit superior behavior compared to joints without all four faces confined by beams under reversed cyclic loading\citep{hanson1967seismic}. 

\cite{najafgholipour2022} evaluated reinforced concrete slab-column connection behavior under simultaneous gravity and lateral loads through a finite element study with ABAQUS to examine flexural reinforcement, slab thickness and supporting column dimensions effects on connection lateral drift capacity for an interior slab-column connection.

